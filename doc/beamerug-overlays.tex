% Copyright 2003--2007 by Till Tantau
% Copyright 2010 by Vedran Mileti\'c
%
% This file may be distributed and/or modified
%
% 1. under the LaTeX Project Public License and/or
% 2. under the GNU Free Documentation License.
%
% See the file doc/licenses/LICENSE for more details.

% $Header$

\section{Creating Overlays}

\label{section-overlay}



\subsection{The Pause Commands}

The |pause| command offers an easy, but not very flexible
way of creating frames that are uncovered piecewise. If you say
|\pause| somewhere in a frame, only the text on the frame up to the
|\pause| command is shown on the first slide. On the
second slide, everything is shown up to the second |\pause|, and
so forth. You can also use |\pause| inside environments; its effect
will last after the environment. However, taking this to
extremes and using |\pause| deeply within a nested environment may not
have the desired result.

A much more fine-grained control over what is shown on each slide can
be attained using overlay specifications, see the next
sections. However, for many simple cases the |\pause|
command is sufficient.

The effect of |\pause| lasts till the next |\pause|, |\onslide|, or
the end of the frame.

\begin{verbatim}
\begin{frame}
  \begin{itemize}
  \item
    Shown from first slide on.
  \pause
  \item
    Shown from second slide on.
    \begin{itemize}
    \item
      Shown from second slide on.
    \pause
    \item
      Shown from third slide on.
    \end{itemize}
  \item
    Shown from third slide on.
  \pause
  \item
    Shown from fourth slide on.
  \end{itemize}

  Shown from fourth slide on.

  \begin{itemize}
  \onslide
  \item
    Shown from first slide on.
  \pause
  \item
    Shown from fifth slide on.
  \end{itemize}
\end{frame}
\end{verbatim}

\begin{command}{\pause\oarg{number}}
  This command causes the text following it to be shown only from the
  next slide on, or, if the optional \meta{number} is given,
  from the slide with the number \meta{number}. If the optional
  \meta{number} is given, the counter |beamerpauses| is set to this
  number. This command uses the |\onslide| command, internally.
  This command does \emph{not} work inside |amsmath| environments like
  |align|, since these do really wicked things.

  \example
\begin{verbatim}
\begin{frame}
  \begin{itemize}
  \item
    A
  \pause
  \item
    B
  \pause
  \item
    C
  \end{itemize}
\end{frame}
\end{verbatim}

  \articlenote
  This command is ignored.

  \lyxnote
  Use the ``Pause'' style with an empty line to insert a pause.
\end{command}

To ``unpause'' some text, that is, to temporarily suspend pausing, use
the command |\onslide|, see below.






\subsection{The General Concept of Overlay Specifications}

\label{section-concept-overlays}

The approach taken by most presentation classes to overlays is
somewhat similar to the above |\pause| command. These commands get a
certain slide number as input and affect the text on the slide
following this command in a certain way. For example,
\textsc{prosper}'s |\FromSlide{2}| command causes all text following
this command to be shown only from the second slide on.

The \beamer\ class uses a different approach (though the
abovementioned command is also available: |\onslide<2->| will have the
same effect as |\FromSlide{2}|, except that |\onslide| transcends
environments; likewise, |\pause| is internally mapped to a command
with an appropriate overlay specifications). The idea is to add
\emph{overlay specifications} to certain commands. These
specifications are always given in pointed
brackets and follow the command ``as soon as possible,'' though in
certain cases \beamer\ also allows overlay specification to be
given a little later. In the simplest case, the specification contains
just a number. A command with an overlay
specification following it will only have ``effect'' on the slide(s)
mentioned in the specification. What exactly ``having an effect''
means, depends on the command. Consider the following example.

\begin{verbatim}
\begin{frame}
  \textbf{This line is bold on all three slides.}
  \textbf<2>{This line is bold only on the second slide.}
  \textbf<3>{This line is bold only on the third slide.}
\end{frame}
\end{verbatim}

For the command |\textbf|, the overlay specification causes the
text to be set in boldface only on the specified slides. On all other
slides, the text is set in a normal font.

For a second example, consider the following frame:
\begin{verbatim}
\begin{frame}
  \only<1>{This line is inserted only on slide 1.}
  \only<2>{This line is inserted only on slide 2.}
\end{frame}
\end{verbatim}

The command |\only|, which is introduced by \beamer, normally simply
inserts its parameter into the current frame. However, if an
overlay-specification is present, it ``throws away'' its parameter on
slides that are not mentioned.

Overlay specifications can only be written behind certain commands,
not every command. Which commands you can use and which effects this
will have is explained in the next section. However, it is quite
easy to redefine an existing command such that it becomes ``overlay
specification aware,'' see also
Section~\ref{section-overlay-commands}.

The syntax of (basic) overlay specifications is the following: They
are comma-separated lists of slides and ranges. Ranges are specified
like this: |2-5|, which means slide two through to five. The start or
the end of a range can be omitted. For example, |3-| means
``slides three, four, five, and so on'' and |-5| means the same as
|1-5|. A complicated example is |-3,6-8,10,12-15|, which selects the
slides 1, 2, 3, 6, 7, 8, 10, 12, 13, 14, and 15.


\lyxnote
Overlay specifications can also be given in \LyX. You must give them
in \TeX-mode (otherwise the pointed brackets may be ``escaped'' by
\LyX, though this will not happen in all versions). For example, to
add an overlay specification to an item, simply insert a \TeX-mode
text like |<3>| as the first thing in that item. Likewise, you can add
an overlay specification to environments like |theorem| by giving
them in \TeX-mode right at the start of the environment.


\subsection{Commands with Overlay Specifications}
\label{section-overlay-commands}

For the following commands, adding an overlay specification causes the
command to be simply ignored on slides that are not included in the
specification: |\textbf|, |\textit|, |\textsl|,
|\textrm|, |\textsf|, |\color|, |\alert|,
|\structure|. If a command takes several arguments, like
|\color|, the specification should directly follow the command as in
the following example (but there are exceptions to this rule):
\begin{verbatim}
\begin{frame}
  \color<2-3>[rgb]{1,0,0} This text is red on slides 2 and 3, otherwise black.
\end{frame}
\end{verbatim}

For the following commands, the effect of an overlay specification is
special:

\begin{command}{\onslide\opt{\meta{modifier}}\sarg{overlay specification}\opt{\marg{text}}}
  The behaviour of this command depends on whether the optional
  argument \meta{text} is given or not (note that the optional
  argument is given in \emph{normal} braces, not in square
  brackets). If present, the \meta{modifier} can be either a~|+| or
  a~|*|.

  If no \meta{text} is given, the following happens: All text
  following this command will only be shown  (uncovered) on the
  specified slides. On non-specified slides, the text still
  occupies space. If no slides are specified, the following
  text is always shown. You need not call this command in the same
  \TeX\ group, its effect transcends block groups. However, this
  command has a \emph{different} effect inside an |overprint|
  environment, see the description of |overprint|.

  If the \meta{modifier} is |+|, hidden text will not be treated
  as covered, but as invisible. The difference is the same as the
  difference between |\uncover| and |\visible|. The modifier |*| may
  not be given if no \meta{text} argument is present.

  \example
\begin{verbatim}
\begin{frame}
  Shown on first slide.
  \onslide<2-3>
  Shown on second and third slide.
  \begin{itemize}
  \item
    Still shown on the second and third slide.
  \onslide+<4->
  \item
    Shown from slide 4 on.
  \end{itemize}
  Shown from slide 4 on.
  \onslide
  Shown on all slides.
\end{frame}
\end{verbatim}

  If a \meta{text} argument is present, |\onslide| (without a
  \meta{modifier}) is mapped to |\uncover|, |\onslide+|
  is mapped to |\visible|, and |\onslide*| is mapped to |\only|.

  \example
\begin{verbatim}
\begin{frame}
  \onslide<1>{Same effect as the following command.}
  \uncover<1>{Same effect as the previous command.}

  \onslide+<2>{Same effect as the following command.}
  \visible<2>{Same effect as the previous command.}

  \onslide*<3>{Same effect as the following command.}
  \only<3>{Same effect as the previous command.}
\end{frame}
\end{verbatim}
\end{command}


\begin{command}{\only\sarg{overlay
      specification}\marg{text}\sarg{overlay specification}}
  If either \meta{overlay specification} is present (though only one
  may be present), the \meta{text} is inserted only into the specified
  slides. For other slides, the text is simply thrown away. In
  particular, it occupies no space.

  \example |\only<3->{Text inserted from slide 3 on.}|

  Since the overlay specification may also be given after the text,
  you can often use |\only| to make other commands
  overlay-specification-aware in a simple manner:

  \example
\begin{verbatim}
\newcommand{\myblue}{\only{\color{blue}}}
\begin{frame}
  \myblue<2> This text is blue only on slide 2.
\end{frame}
\end{verbatim}
\end{command}


\begin{command}{\uncover\sarg{overlay specification}\marg{text}}
  If the \meta{overlay specification} is present, the \meta{text} is
  shown (``uncovered'') only on the specified slides. On other slides, the
  text still occupies space and it is still typeset, but it is not
  shown or only shown as if transparent. For details on how to specify
  whether the text is invisible or just transparent see
  Section~\ref{section-transparent}.
  \example |\uncover<3->{Text shown from slide 3 on.}|

  \articlenote
  This command has the same effect as |\only|.
\end{command}

\begin{command}{\visible\sarg{overlay specification}\marg{text}}
  This command does almost the same as |\uncover|. The only difference
  is that if the text is not shown, it is never shown in a transparent
  way, but rather it is not shown at all. Thus, for this command the
  transparency settings have no effect.

  \example |\visible<2->{Text shown from slide 2 on.}|

  \articlenote
  This command has the same effect as |\only|.
\end{command}

\begin{command}{\invisible\sarg{overlay specification}\marg{text}}
  This command is the opposite of |\visible|.

  \example |\invisible<-2>{Text shown from slide 3 on.}|
\end{command}

\begin{command}{\alt\sarg{overlay specification}%
    \marg{default text}\marg{alternative text}\sarg{overlay specification}}
  Only one \meta{overlay specification} may be given.
  The default text is shown on the specified slides, otherwise the
  alternative text. The specification must always be present.
  \example |\alt<2>{On Slide 2}{Not on slide 2.}|

  Once more, giving the overlay specification at the end is useful
  when the command is used inside other commands.

  \example Here is the definition of |\uncover|:
\begin{verbatim}
\newcommand{\uncover}{\alt{\@firstofone}{\makeinvisible}}
\end{verbatim}
\end{command}

\begin{command}{\temporal\ssarg{overlay specification}%
    \marg{before slide text}\marg{default text}\marg{after slide text}}
  This command alternates between three different texts, depending on
  whether the current slide is temporally before the specified
  slides, is one of the specified slides, or comes after them. If the
  \meta{overlay specification} is not an interval (that is, if it has
  a ``hole''), the ``hole'' is considered to be part of the before slides.
  \example
\begin{verbatim}
  \temporal<3-4>{Shown on 1, 2}{Shown on 3, 4}{Shown 5, 6, 7, ...}
  \temporal<3,5>{Shown on 1, 2, 4}{Shown on 3, 5}{Shown 6, 7, 8, ...}
\end{verbatim}

  As a possible application of the |\temporal| command consider the
  following example:
  \example
\begin{verbatim}
\def\colorize<#1>{%
  \temporal<#1>{\color{red!50}}{\color{black}}{\color{black!50}}}

\begin{frame}
  \begin{itemize}
    \colorize<1> \item First item.
    \colorize<2> \item Second item.
    \colorize<3> \item Third item.
    \colorize<4> \item Fourth item.
  \end{itemize}
\end{frame}
\end{verbatim}
\end{command}


\begin{command}{\item\sarg{alert specification}\oarg{item
      label}\sarg{alert specification}}
  \beamernote
  Only one \meta{alert specification} may be given. The effect of
  \meta{alert specification} is described in
  Section~\ref{section-action-specifications}.

  \example
\begin{verbatim}
\begin{frame}
  \begin{itemize}
  \item<1-> First point, shown on all slides.
  \item<2-> Second point, shown on slide 2 and later.
  \item<2-> Third point, also shown on slide 2 and later.
  \item<3-> Fourth point, shown on slide 3.
  \end{itemize}
\end{frame}

\begin{frame}
  \begin{enumerate}
  \item<3-| alert@3>[0.] A zeroth point, shown at the very end.
  \item<1-| alert@1> The first and main point.
  \item<2-| alert@2> The second point.
  \end{enumerate}
\end{frame}
\end{verbatim}

  \articlenote
  The \meta{action specification} is currently completely ignored.

  \lyxnote
  The \meta{action specification} must be given in \TeX-mode and it
  must be given at the very start of the item.
\end{command}

The related command |\bibitem| is also overlay-specification-aware
in the same way as |\item|.

\begin{command}{\label\sarg{overlay specification}\marg{label name}}
  If the \meta{overlay specification} is present, the label is only
  inserted on the specified slide. Inserting a label on more than one
  slide will cause a `multiple labels' warning. \emph{However}, if no
  overlay specification is present, the specification is automatically
  set to just `1' and the label is thus inserted only on the first
  slide. This is typically the desired behaviour since it does not
  really matter on which slide the label is inserted, \emph{except} if
  you use an |\only| command and \emph{except} if you wish to use that
  label as a hyperjump target. Then you need to specify a slide.

  Labels can be used as target of hyperjumps. A convenient way of
  labelling a frame is to use the |label=|\meta{name} option of the
  |frame| environment. However, this will cause the whole frame to be
  kept in memory till the end of the compilation, which may pose a
  problem.
  \example
\begin{verbatim}
\begin{frame}
  \begin{align}
    a &= b + c   \label{first}\\ % no specification needed
    c &= d + e   \label{second}\\% no specification needed
  \end{align}

  Blah blah, \uncover<2>{more blah blah.}

  \only<3>{Specification is needed now.\label<3>{mylabel}}
\end{frame}
\end{verbatim}
\end{command}



\subsection{Environments with Overlay Specifications}

Environments can also be equipped with overlay specifications. For
most of the predefined environments, see Section~\ref{predefined},
adding an overlay specification causes the whole environment to be
uncovered only on the specified slides. This is useful for showing
things incrementally as in the following example.

\begin{verbatim}
\begin{frame}
  \frametitle{A Theorem on Infinite Sets}

  \begin{theorem}<1->
    There exists an infinite set.
  \end{theorem}

  \begin{proof}<3->
    This follows from the axiom of infinity.
  \end{proof}

  \begin{example}<2->
    The set of natural numbers is infinite.
  \end{example}
\end{frame}
\end{verbatim}
In the example, the first slide only contains the theorem, on the
second slide an example is added, and on the third slide the proof is
also shown.

For each of the basic commands |\only|, |\alt|, |\visible|,
|\uncover|, and |\invisible| there exists
``environment versions'' |onlyenv|, |altenv|, |visibleenv|,
|uncoverenv|, and |invisibleenv|. Except for |altenv|
and |onlyenv|, these environments do the same as the commands.

\begin{environment}{{onlyenv}\sarg{overlay specification}}
  If the \meta{overlay specification} is given, the contents of the
  environment is inserted into the text only on the specified
  slides. The difference to |\only| is, that the text is actually
  typeset inside a box that is then thrown away, whereas |\only|
  immediately throws away its contents. If the text is not
  ``typesettable,'' the |onlyenv| may produce an error where |\only|
  would not.
  \example
\begin{verbatim}
\begin{frame}
  This line is always shown.
  \begin{onlyenv}<2>
    This line is inserted on slide 2.
  \end{onlyenv}
\end{frame}
\end{verbatim}
\end{environment}


\begin{environment}{{altenv}\sarg{overlay specification}\marg{begin
text}\marg{end text}\marg{alternate begin text}\marg{alternate end
text}\sarg{overlay specification}}
  Only one \meta{overlay specification} may be given. On the specified
  slides, \meta{begin text} will be inserted at the beginning of the
  environment and \meta{end text} will be inserted at the end. On all
  other slides, \meta{alternate begin text} and \meta{alternate end
    text} will be used.

  \example
\begin{verbatim}
\begin{frame}
  This
  \begin{altenv}<2>{(}{)}{[}{]}
    word
  \end{uncoverenv}
  is in round brackets on slide 2 and in square brackets on slide 1.
\end{frame}
\end{verbatim}
\end{environment}


\subsection{Dynamically Changing Text or Images}

You may sometimes wish to have some part of a frame change dynamically
from slide to slide. On each slide of the frame, something different
should be shown inside this area. You could achieve the effect of
dynamically changing text by giving a list of |\only| commands like this:
\begin{verbatim}
  \only<1>{Initial text.}
  \only<2>{Replaced by this on second slide.}
  \only<3>{Replaced again by this on third slide.}
\end{verbatim}
The trouble with this approach is that it may lead to slight, but
annoying differences in the heights of the lines, which may cause the
whole frame to ``whobble'' from slide to slide. This problem becomes
much more severe if the replacement text is several lines long.

To solve this problem, you can use two environments:
|overlayarea| and |overprint|. The first is more flexible,
but less user-friendly.

\begin{environment}{{overlayarea}\marg{area width}\marg{area height}}
  Everything within the environment will be placed in a rectangular
  area of the specified size. The area will have the same size on all
  slides of a frame, regardless of its actual contents.
  \example
\begin{verbatim}
\begin{overlayarea}{\textwidth}{3cm}
  \only<1>{Some text for the first slide.\\Possibly several lines long.}
  \only<2>{Replacement on the second slide.}
\end{overlayarea}
\end{verbatim}

  \lyxnote
  Use the style ``OverlayArea'' to insert an overlay area.
\end{environment}

\begin{environment}{{overprint}\oarg{area width}}
  The \meta{area width} defaults to the text width.
  Inside the environment, use |\onslide| commands to specify
  different things that should be shown for this environment on
  different slides. The |\onslide| commands are used like
  |\item| commands. Everything within the environment will be
  placed in a rectangular area of the specified width. The height and
  depth of the area are chosen large enough to accommodate the largest
  contents of the area. The overlay specifications of the
  |\onslide| commands must be disjoint. This may be a problem for
  handouts, since, there, all overlay specifications default to |1|. If
  you use the option |handout|, you can disable all but one
  |\onslide| by setting the others to |0|.
  \example
\begin{verbatim}
\begin{overprint}
  \onslide<1| handout:1>
    Some text for the first slide.\\
    Possibly several lines long.
  \onslide<2| handout:0>
    Replacement on the second slide. Supressed for handout.
\end{overprint}
\end{verbatim}

  \lyxnote
  Use the style ``Overprint'' to insert an |overprint|
  environment. You have to use \TeX-mode to insert the |\onslide|
  commands.
\end{environment}


A similar need for dynamical changes arises when you have, say, a
series of pictures named |first.pdf|, |second.pdf|, and |third.pdf|
that show different stages of some process. To make a frame that shows
these pictures on different slides, the following code might be used:

\begin{verbatim}
\begin{frame}
  \frametitle{The Three Process Stages}

  \includegraphics<1>{first.pdf}
  \includegraphics<2>{second.pdf}
  \includegraphics<3>{third.pdf}
\end{frame}
\end{verbatim}

The above code uses the fact the \beamer\ makes the |\includegraphics|
command overlay-specification-aware. It works nicely, but only if each
|.pdf| file contains the complete graphic to be shown. However, some
programs, like |xfig|, sometimes also produce series of graphics in
which each file just contains the \emph{additional} graphic elements
to be shown on the next slide. In this case, the first graphic must be
shown not on overlay~1, but from overlay~1 on, and so on. While this
is easy to achieve by changing the overlay specification |<1>| to
|<1->|, the graphics must also be shown \emph{on top of each
  other}. An easy way to achieve this is to use \TeX's |\llap|
command like this:

\begin{verbatim}
\begin{frame}
  \frametitle{The Three Process Stages}

  \includegraphics<1->{first.pdf}%
  \llap{\includegraphics<2->{second.pdf}}%
  \llap{\includegraphics<3->{third.pdf}}
\end{frame}
\end{verbatim}

or like this:

\begin{verbatim}
\begin{frame}
  \frametitle{The Three Process Stages}

  \includegraphics{first.pdf}%
  \pause%
  \llap{\includegraphics{second.pdf}}%
  \pause%
  \llap{\includegraphics{third.pdf}}
\end{frame}
\end{verbatim}

A more convenient way is to use the |\multiinclude| command, see
Section~\ref{section-xmpmulti} for details.




\subsection{Advanced Overlay Specifications}

\subsubsection{Making Commands and Environments Overlay-Specification-Aware}

This section explains how to define new commands that are
overlay-specification-aware. Also, it explains how to setup counters
correctly that should be increased from frame to frame (like equation
numbering), but not from slide to slide. You may wish to skip this
section, unless you  want to write your own extensions to the \beamer\
class.

\beamer\ extends the syntax of \LaTeX's standard command
|\newcommand|:


\begin{command}{\newcommand\declare{|<>|}\marg{command name}%
    \oarg{argument number}\oarg{default optional value}\marg{text}}
  Declares the new command named \meta{command name}. The \meta{text}
  should contain the body of this command and it may contain
  occurrences of parameters like |#|\meta{number}. Here \meta{number}
  may be between 1 and $\mbox{\meta{argument number}}+1$. The
  additionally allowed argument is the overlay specification.

  When \meta{command name} is used, it will scan as many as
  \meta{argument number} arguments. While scanning them, it will look
  for an overlay specification, which may be given between any two
  arguments, before the first argument, or after the last argument. If
  it finds an overlay specification like |<3>|, it will call
  \meta{text} with arguments 1 to \meta{argument number} set to the
  normal arguments and the argument number $\mbox{\meta{argument
      number}}+1$ set to |<3>| (including the pointed brackets). If no
  overlay specification is found, the extra argument is empty.

  If the \meta{default optional value} is provided, the first argument
  of \meta{command name} is optional. If no optional argument is
  specified in square brackets, the \meta{default optional value} is
  used.

  \example The following command will typeset its argument in red on
  the specified slides:
\begin{verbatim}
\newcommand<>{\makered}[1]{{\color#2{red}#1}}
\end{verbatim}

  \example Here is \beamer's definition of |\emph|:
\begin{verbatim}
\newcommand<>{\emph}[1]{{\only#2{\itshape}#1}}
\end{verbatim}


  \example Here is \beamer's definition of |\transdissolve| (the
  command |\beamer@dotrans| mainly passes its argument to |hyperref|):
\begin{verbatim}
\newcommand<>{\transdissolve}[1][]{\only#2{\beamer@dotrans[#1]{Dissolve}}}
\end{verbatim}
\end{command}

\begin{command}{\renewcommand\declare{|<>|}\marg{existing command name}%
    \oarg{argument number}\oarg{default optional value}\marg{text}}
  Redeclares a command that already exists in the same way as
  |\newcommand<>|. Inside \meta{text}, you can
  still access to original definitions using the command
  |\beameroriginal|, see the example.
  \example This command is used in \beamer\ to make |\hyperlink| overlay-specification-aware:
\begin{verbatim}
\renewcommand<>{\hyperlink}[2]{\only#3{\beameroriginal{\hyperlink}{#1}{#2}}}
\end{verbatim}
\end{command}


\begin{command}{\newenvironment\declare{|<>|}\marg{environment name}%
    \oarg{argument number}\oarg{default optional value}\\\marg{begin
    text}\marg{end text}}
  Declares a new environment that is overlay-specification-aware. If
  this environment is encountered, the same algorithm as for
  |\newcommand<>| is used to parse the arguments and the overlay
  specification.

  Note that, as always, the \meta{end text} may not contain any
  arguments like |#1|. In particular, you do not have access to the
  overlay specification. In this case, it is usually a good idea to
  use |altenv| environment in the \meta{begin text}.

  \example Declare your own action block:
\begin{verbatim}
\newenvironment<>{myboldblock}[1]{%
  \begin{actionenv}#2%
    \textbf{#1}
    \par}
  {\par%
  \end{actionenv}}

\begin{frame}
  \begin{myboldblock}<2>
    This theorem is shown only on the second slide.
  \end{myboldblock}
\end{frame}
\end{verbatim}

  \example Text in the following environment is normally bold and
  italic on non-specified slides:
\begin{verbatim}
\newenvironment<>{boldornormal}
  {\begin{altenv}#1
    {\begin{bfseries}}{\end{bfseries}}
    {}{}}
  {\end{altenv}}
\end{verbatim}

  Incidentally, since |altenv| also accepts its argument at the end,
  the same effect could have been achieved using just
  \begin{verbatim}
\newenvironment{boldornormal}
  {\begin{altenv}
    {\begin{bfseries}}{\end{bfseries}}
    {}{}}
  {\end{altenv}}
\end{verbatim}
\end{command}

\begin{command}{\renewenvironment\declare{|<>|}\marg{existing environment name}%
    \oarg{argument number}\oarg{default optional value}\\
    \marg{begin
    text}\marg{end text}}
  Redefines an existing environment. The original environment is still
  available under the name |original|\meta{existing environment name}.

  \example
\begin{verbatim}
\renewenvironment<>{verse}
{\begin{actionenv}#1\begin{originalverse}}
{\end{originalverse}\end{actionenv}}
\end{verbatim}
\end{command}

The following two commands can be used to ensure that a certain
counter is automatically reset on subsequent slides of a frame. This
is necessary for example for the equation count. You might want this
count to be increased from frame to frame, but certainly not from
overlay slide to overlay slide. For equation counters and footnote
counters (you should not use footnotes), these commands have already
been invoked.

\begin{command}{\resetcounteronoverlays\marg{counter name}}
  After you have invoked this command, the value of the specified
  counter will be the same on all slides of every frame.
  \example |\resetcounteronoverlays{equation}|
\end{command}

\begin{command}{\resetcountonoverlays\marg{count register name}}
  The same as |\resetcounteronoverlays|, except that this
  command should be used with counts that have been created using the
  \TeX\ primitive |\newcount| instead of \LaTeX's  |\definecounter|.
  \example
\begin{verbatim}
\newcount\mycount
\resetcountonoverlays{mycount}
\end{verbatim}
\end{command}






\subsubsection{Mode Specifications}

This section is only important if you use \beamer's mode mechanism
to create different versions of your presentation. If you are not
familiar with \beamer's modes, please skip this section or read
Section~\ref{section-modes} first.

In certain cases you may wish to have different overlay specifications
to apply to a command in different modes.
For example, you might wish a certain text to appear only from the
third slide on during your presentation, but in a handout for the
audience there should be no second slide and the text should appear
already on the second slide. In this case you could write
\begin{verbatim}
\only<3| handout:2>{Some text}
\end{verbatim}

The vertical bar, which must be followed by a (white) space, separates
the two different specifications |3| and |handout:2|. By writing a
mode name before a colon, you specify that the following specification
only applies to that mode. If no mode is given, as in |3|, the mode
|beamer| is automatically added. For this reason, if you write
|\only<3>{Text}| and you are in |handout| mode, the text will be shown
on all slides since there is no restriction specified for handouts and
since the |3| is the same as |beamer:3|.

It is also possible to give an overlay specification that contains
only a mode name (or several, separated by vertical bars):
\begin{verbatim}
\only<article>{This text is shown only in article mode.}
\end{verbatim}
An overlay specification that does not contain any slide numbers is
called a (pure) \emph{mode specification}. If a mode specification is
given, all modes that are not mentioned are automatically
suppressed. Thus |<beamer:1->| means ``on all slides in |beamer| mode
and also on all slides in all other modes, since nothing special is
specified for them,'' whereas |<beamer>| means ``on all slides in
|beamer| mode and not on any other slide.''

Mode specifications can also be used outside frames as in the following
examples:
\begin{verbatim}
\section<presentation>{This section exists only in the presentation modes}
\section<article>{This section exists only in the article mode}
\end{verbatim}

You can also mix pure mode specifications and overlay specifications,
although this can get confusing:
\begin{verbatim}
\only<article| beamer:1>{Riddle}
\end{verbatim}

This will cause the text |Riddle| to be inserted in |article| mode and
on the first slide of a frame in |beamer| mode, but not at all in
|handout| or |trans| mode. (Try to find out how
\verb/<beamer| beamer:1>/ differs from |<beamer>| and from
|<beamer:1>|.)

As if all this were not already complicated enough, there is another
mode that behaves in a special way: the mode |second|. For this mode a
special rule applies: An overlay specification for mode |beamer| also
applies to mode |second| (but not the other way round). Thus, if we
are in mode |second|, the specification |<second:2>| means ``on slide
2'' and |<beamer:2>| also means ``on slide 2''. To get a slide that is
typeset in |beamer| mode, but not in |second| mode, you can use,
|<second:0>|.



\subsubsection{Action Specifications}
\label{section-action-specifications}

This section also introduces a rather advanced concept. You may
also wish to skip it on first reading.

Some overlay-specification-aware commands cannot only handle normal
overlay specifications, but also so called \emph{action
specifications}. In an action specification, the list of slide numbers
and ranges is prefixed by \meta{action}|@|, where \meta{action} is the
name of a certain action to be taken on the specified slides:
\begin{verbatim}
\item<3-| alert@3> Shown from slide 3 on, alerted on slide 3.
\end{verbatim}
In the above example, the |\item| command, which allows actions to be
specified, will uncover the item text from slide three on and it will,
additionally, alert this item exactly on slide 3.

Not all commands can take an action specification. Currently, only
|\item| (though not in |article| mode currently), |\action|, the
environment |actionenv|, and the block environments (like |block| or
|theorem|) handle them.

By default, the following actions are available:
\begin{itemize}
\item \declare{|alert|} alters the item or block.
\item \declare{|uncover|} uncovers the item or block (this is
  the default, if no action is specified).
\item \declare{|only|} causes the whole item or block
  to be inserted only on the specified slides.
\item \declare{|visible|} causes the text to become visible only on
  the specified slides (the difference between |uncover| and
  |visible| is the same as between |\uncover| and |\visible|).
\item \declare{|invisible|} causes the text to become invisible on the
  specified slides.
\end{itemize}

The rest of this section explains how you can add your own actions and
make commands action-specification-aware. You may wish to skip it upon
first reading.

You can easily add your own actions: An action specification like
\meta{action}|@|\meta{slide numbers} simply inserts an environment
called \meta{action}|env| around the |\item| or parameter of
|\action| with |<|\meta{slide numbers}|>| as overlay
specification. Thus, by defining a new overlay-specification-aware
environment named \meta{my action name}|env|, you can add your own
action:
\begin{verbatim}
\newenvironment{checkenv}{\only{\setbeamertemplate{itemize item}{X}}}{}
\end{verbatim}
You can then  write
\begin{verbatim}
\item<beamer:check@2> Text.
\end{verbatim}
This will change the itemization symbol before |Text.| to |X| on
slide~2 in |beamer| mode. The definition of |checkenv| used the fact
that |\only| also accepts an overlay-specification given after its
argument.

The whole action mechanism is based on the following environment:

\begin{environment}{{actionenv}\sarg{action specification}}
  This environment extracts all actions from the \meta{action
    specification} for the current mode. For each action of the form
  \meta{action}|@|\meta{slide numbers}, it inserts the following text:
  |\begin{|\meta{action}|env}<|\meta{slide number}|>| at the
  beginning of the environment and the text |\end{|\meta{action}|env}|
  at the end. If there are several action specifications, several
  environments are opened (and closed in the appropriate order). An
  \meta{overlay specification} without an action is promoted to
  |uncover@|\meta{overlay specification}.

  If the so called \emph{default overlay specification} is not empty,
  it will be used in case no \meta{action specification} is given. The
  default overlay specification is usually just empty, but it may be
  set either by providing an additional optional argument to the
  command |\frame| or to the environments |itemize|, |enumerate|, or
  |description| (see these for details). Also, the default action
  specification can be set using the command
  |\beamerdefaultoverlayspecification|, see below.

  \example
\begin{verbatim}
\begin{frame}
  \begin{actionenv}<2-| alert@3-4,6>
    This text is shown the same way as the text below.
  \end{actionenv}

  \begin{uncoverenv}<2->
    \begin{alertenv}<3-4,6>
      This text is shown the same way as the text above.
    \end{alertenv}
  \end{uncoverenv}
\end{frame}
\end{verbatim}
\end{environment}

\begin{command}{\action\sarg{action specification}\marg{text}}
  This has the same effect as putting \meta{text} in an |actionenv|.

  \example |\action<alert@2>{Could also have used \alert<2>{}.}|
\end{command}

\begin{command}{\beamerdefaultoverlayspecification\marg{default
      overlay specification}}
  Locally sets the default overlay specification to the given
  value. This overlay specification will be used in every |actionenv|
  environment and every |\item| that does not have its own overlay
  specification. The main use of this command is to install an
  incremental overlay specification like |<+->| or
  \verb/<+-| alert@+>/, see Section~\ref{section-incremental}.

  Usually, the default overlay specification is installed
  automatically by the optional arguments to |\frame|, |frame|,
  |itemize|, |enumerate|, and |description|. You will only have to use
  this command if you wish to do funny things.

  If given outside any frame, this command sets the default overlay
  specification for all following frames for which you do not override
  the default overlay specification.

  \example |\beamerdefaultoverlayspecification{<+->}|

  \example |\beamerdefaultoverlayspecification{}| clears the default
  overlay specification. (Actually, it installs the default overlay
  specification |<*>|, which just means ``always,'' but the
  ``portable'' way of clearing the default overlay specification is
  this call.)
\end{command}



\subsubsection{Incremental Specifications}
\label{section-incremental}

This section is mostly important for people who have already used
overlay specifications a lot and have grown tired of writing things
like |<1->|, |<2->|, |<3->|, and so on again and again. You should
skip this section on first reading.

Often you want to have overlay specifications that follow a pattern
similar to the following:
\begin{verbatim}
\begin{itemize}
\item<1-> Apple
\item<2-> Peach
\item<3-> Plum
\item<4-> Orange
\end{itemize}
\end{verbatim}
The problem starts if you decide to insert a new fruit, say, at the
beginning. In this case, you would have to adjust all of the overlay
specifications. Also, if you add a |\pause| command before the
|itemize|, you would also have to update the overlay specifications.

\beamer\ offers a special syntax to make creating lists as the one
above more ``robust.'' You can replace it by the following list of
\emph{incremental overlay specifications}:
\begin{verbatim}
\begin{itemize}
\item<+-> Apple
\item<+-> Peach
\item<+-> Plum
\item<+-> Orange
\end{itemize}
\end{verbatim}
The effect of the |+|-sign is the following: You can use it in any
overlay specification at any point where you would usually use a
number. If a |+|-sign is encountered, it is replaced by the current
value of the \LaTeX\ counter |beamerpauses|, which is 1 at the
beginning of the frame. Then the counter is increased by 1, though it
is only increased once for every overlay specification, even if the
specification contains multiple |+|-signs (they are replaced by the
same number).

In the above example, the first specification is replaced by
|<1->|. Then the second is replaced by |<2->| and so forth. We can now
easily insert new entries, without having to change anything. We might
also write the following:
\begin{verbatim}
\begin{itemize}
\item<+-| alert@+> Apple
\item<+-| alert@+> Peach
\item<+-| alert@+> Plum
\item<+-| alert@+> Orange
\end{itemize}
\end{verbatim}
This will alert the current item when it is uncovered. For example,
the first specification \verb/<+-| alert@+>/ is replaced by
\verb/<1-| alert@1>/, the second is replaced by \verb/<2-| alert@2>/, and so on.
Since the |itemize| environment also allows you to specify a default
overlay specification, see the documentation of that environment, the
above example can be written even more economically as follows:
\begin{verbatim}
\begin{itemize}[<+-| alert@+>]
\item Apple
\item Peach
\item Plum
\item Orange
\end{itemize}
\end{verbatim}

The |\pause| command also updates the counter |beamerpauses|. You can
change this counter yourself using the normal \LaTeX\ commands
|\setcounter| or |\addtocounter|.

Any occurence of a |+|-sign may be followed by an \emph{offset} in
round brackets. This offset will be added to the value of
|beamerpauses|. Thus, if |beamerpauses| is 2, then |<+(1)->| expands to
|<3->| and |<+(-1)-+>| expands to |<1-2>|.

There is another special sign you can use in an overlay specification
that behaves similarly to the |+|-sign: a dot. When you write |<.->|,
a similar thing as in |<+->| happens \emph{except that the counter
  |beamerpauses| is not incremented} and \emph{except that you get the
  value of |beamerpauses| decreased by one}. Thus a dot, possibly
followed by an offset, just expands to the current value of the
counter |beamerpauses| minus one, possibly offset. This dot notation
can be useful in case like the following:
\begin{verbatim}
\begin{itemize}[<+->]
\item Apple
\item<.-> Peach
\item Plum
\item Orange
\end{itemize}
\end{verbatim}
In the example, the second item is shown at the same time as the first
one since it does not update the counter.

In the following example, each time an item is uncovered, the
specified text is alerted. When the next item is uncovered, this
altering ends.
\begin{verbatim}
\begin{itemize}[<+->]
\item This is \alert<.>{important}.
\item We want to \alert<.>{highlight} this and \alert<.>{this}.
\item What is the \alert<.>{matrix}?
\end{itemize}
\end{verbatim}
